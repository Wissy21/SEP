\chapter{Nicht-funktionale Anforderungen}

\newcounter{nf}\setcounter{nf}{10}

\section{Softwarearchitektur}

\begin{description}[leftmargin=5em, style=sameline]	
	\begin{lhp}{nf}{NF}{nfunk:sarch1}
		\item [Name:] Client-Server Anwendung
		\item [Beschreibung:] Das verteilte Spiele-System ermöglicht das gemeinsame Spielen von verschiedenen Rechnern aus.
		\item [Motivation:] Aufgabestellung v. SEP/MP.
		\item [Erfüllungskriterium:] Das fertige System besteht aus Client- und Server-Teilen.
	\end{lhp}
	
	\begin{lhp}{nf}{NF}{nfunk:sarch2}
		\item [Name:] Plattformunabhängigkeit
		\item [Beschreibung:] Es soll sich um eine plattformunabhängige Anwendung handeln. Zumindest Windows- und Linuxsysteme sind zu unterstützen.
		\item [Motivation:] Aufgabenstellung v. SEP/MP.
		\item [Erfüllungskriterium:] Das Spiel funktioniert auf Linux- und Windowssystemen
	\end{lhp}
\end{description}



\section{Benutzerfreundlichkeit}


\begin{description}[leftmargin=5em, style=sameline]	
	\begin{lhp}{nf}{NF}{nfunk:alter}
		\item [Name:] Benutzeralter
		\item [Beschreibung:] Das System ist für Benutzer geeignet, die älter als 5 Jahre sind.
		\item [Motivation:] Jüngere Benutzer sind novh unfähig das Spiel zu spielen.
		\item [Erfüllungskriterium:] In den AGBs steht ein entsprechender Hinweis.
	\end{lhp}
\end{description}

\begin{description}[leftmargin=5em, style=sameline]	
	\begin{lhp}{nf}{NF}{nfunk:keinetechniker}
		\item [Name:] Technische Fähigkeiten
		\item [Beschreibung:] Besondere technische Fähigkeiten sind von den Benutzern nicht zu erwarten.
		\item [Motivation:] Auch die Menschen, die kaum etwas von Bedienung bzw. Programmierung von Rechnern verstehen, sollen fähig sein, das System zu verwenden.
		\item [Erfüllungskriterium:] Intuitive GUI
	\end{lhp}
\end{description}

\section{Leistungsanforderungen}

\begin{description}[leftmargin=5em, style=sameline]	
	\begin{lhp}{nf}{NF}{nfunk:antwortzeit}
		\item [Name:] Antwortzeit
		\item [Beschreibung:] Maximale Antwortzeit für alle Systemprozesse.
		\item [Motivation:] Das System muss immer brauchbar sein.
		\item [Erfüllungskriterium:] Das System antwortet auf Benutzerhandlungen nie später als in 10 Sekunden.
	\end{lhp}
\end{description}

\section{Anforderungen an Einsatzkontext}

\subsection{Anforderungen an physische Umgebung}

\begin{description}[leftmargin=5em, style=sameline]	
	\begin{lhp}{nf}{NF}{nfunk:beispiel1}
		\item [Name:] Lauffähigkeit an SCI-Rechnern
		\item [Beschreibung:] Das Produkt muss auf einem eigenem Gerät lauffähig sein, welches zur Präsentation am Ende des SEPs genutzt werden muss. Falls keine eigenen Rechner vorhanden sind, stehen auch die SCI-Terminals zur verfügung.
		\item [Motivation:] Optimierung von Betreuung und Abnahme des SEP/MP
		\item [Erfüllungskriterium:] Spiel ist lauffähig auf SCI-Rechner
	\end{lhp}
\end{description}


%\subsection{Anforderungen an benachbarte Systeme}
%(sehe Systemkontext)
%
%\begin{description}[leftmargin=5em, style=sameline]	
%	\begin{lhp}{nf}{NF}{nfunk:beispiel2}
%		\item [Name:] Beispiel
%		\item [Beschreibung:] 
%		\item [Motivation:] 
%		\item [Erfüllungskriterium:] 
%	\end{lhp}
%\end{description}

\subsection{Absatz- sowie Installationsbezogene Anforderungen}

\begin{description}[leftmargin=5em, style=sameline]	
	\begin{lhp}{nf}{NF}{nfunk:beispiel3}
		\item [Name:] Installationsanleitung	
		\item [Beschreibung:] Falls die Installation nicht lediglich das Öffnen einer Datei voraussetz, muss der genaue Installations- und Startvorgang schriftlich für Benutzer zur Verfügung gestellt werden.
		\item [Motivation:] Spezifikation
		\item [Erfüllungskriterium:] 
	\end{lhp}
\end{description}

\subsection{Anforderungen an Versionierung}

\begin{description}[leftmargin=5em, style=sameline]	
	\begin{lhp}{nf}{NF}{nfunk:beispiel4}
		\item [Name:] Keine weitere Versionen
		\item [Beschreibung:] Nach Version 1.0 ist keine weitere Entwicklung vorgesehen.
		\item [Motivation:] Das ist nur das SEP/MP, kein Geschäftsprojekt, siehe \ref{fa:fortentwicklung}
		\item [Erfüllungskriterium:] 
	\end{lhp}
\end{description}

\section{Anforderungen an Wartung und Unterstützung}

\subsection{Wartungsanforderungen}

\begin{description}[leftmargin=5em, style=sameline]	
	\begin{lhp}{nf}{NF}{nfunk:doku1}
		\item [Name:] Dokumentation
		\item [Beschreibung:] Der Quellcode muss ausführlich dokumentiert werden.
		\item [Motivation:] Quellcode soll von jedem verstanden werden, nicht nur vom Ersteller.
		\item [Erfüllungskriterium:] JavaDoc 
	\end{lhp}
\end{description}

\begin{description}[leftmargin=5em, style=sameline]	
	\begin{lhp}{nf}{NF}{nfunk:doku2}
		\item [Name:] Testen
		\item [Beschreibung:] Der Quellcode außer GUI muss gut getestet werden.
		\item [Motivation:] Das Spiel sollte möglichst keine Fehler enthalten
		\item [Erfüllungskriterium:] Von Unit-Tests muss mindestens 70\% des Quellcodes bedeckt werden. GUI-Klassen sind aus der Anforderung ausgenommen.
	\end{lhp}
\end{description}

\subsection{Anforderungen an technische und fachliche Unterstützung}

\begin{description}[leftmargin=5em, style=sameline]	
	\begin{lhp}{nf}{NF}{nfunk:beispiel5}
		\item [Name:] Unterstützung
		\item [Beschreibung:] Es ist keine technische und fachliche Unterstützung des Systems geplant.
		\item [Motivation:] Siehe \ref{fa:fortentwicklung}.
		\item [Erfüllungskriterium:] Nicht anwendbar.
	\end{lhp}
\end{description}

\subsection{Anforderungen an technische Kompatibilität}

\begin{description}[leftmargin=5em, style=sameline]	
	\begin{lhp}{nf}{NF}{nfunk:beispiel6}
		\item [Name:] Kompatibilität
		\item [Beschreibung:] Siehe \ref{nfunk:sarch2}
		\item [Motivation:]
		\item [Erfüllungskriterium:] 
	\end{lhp}
\end{description}

\section{Sicherheitsanforderungen}

\subsection{Zugang}

\begin{description}[leftmargin=5em, style=sameline]	
	\begin{lhp}{nf}{NF}{nfunk:beispiel7}
		\item [Name:] Spielerzugriff
		\item [Beschreibung:] Spieler haben nur Zugriff auf das Konto auf dem sie angemeldet sind und können nur an diesem Änderungen vornehmen.
		\item [Motivation:] Schutz vor Diebstahl von Konten.
		\item [Erfüllungskriterium:] 
	\end{lhp}
\end{description}

\subsection{Integrität}

\begin{description}[leftmargin=5em, style=sameline]	
	\begin{lhp}{nf}{NF}{nfunk:beispiel8}
		\item [Name:] Regelwerk
		\item [Beschreibung:] Die Spielregeln dürfen von keinem Spieler umgangen werden.
		\item [Motivation:] Verhindern von unfairen Vorteilen.
		\item [Erfüllungskriterium:] 
	\end{lhp}
\end{description}

\subsection{Datenschutz/Privatsphäre}

\begin{description}[leftmargin=5em, style=sameline]	
	\begin{lhp}{nf}{NF}{nfunk:beispiel9}
		\item [Name:] Passwörter
		\item [Beschreibung:] Nur der Server und kein Spieler hat Zugriff auf Passwörter.
		\item [Motivation:] Schutz vor Diebstahl von Daten.
		\item [Erfüllungskriterium:] 
	\end{lhp}
\end{description}


\subsection{Virenschutz}

\begin{description}[leftmargin=5em, style=sameline]	
	\begin{lhp}{nf}{NF}{nfunk:beispiel10}
		\item [Name:] Beispiel
		\item [Beschreibung:] 
		\item [Motivation:] 
		\item [Erfüllungskriterium:] 
	\end{lhp}
\end{description}

\section{Prüfungsbezogene Anforderungen}

Anforderungen, die sich auf die Prüfung/Audit vom System von SEP/MP-Tutoren oder von weiteren Instanzen beziehen.


\begin{description}[leftmargin=5em, style=sameline]	
	\begin{lhp}{nf}{NF}{nfunk:beispiel11}
		\item [Name:] Formate der Systemdokumentation
		\item [Beschreibung:] Systemdokumantation muss in 2 Formen geführt werden (wenn anwendbar): Die Ausgangsdateien (\LaTeX, Dateien der Diagrammerstellungssoftware, Dateien der Grafiksoftware usw.) und PDFs.
		\item [Motivation:] Optimierung der SEP/MP-Betreuung.
		\item [Erfüllungskriterium:] Siehe Beschreibung.
	\end{lhp}
\end{description}

\section{Kulturelle und politische Anforderungen}


\begin{description}[leftmargin=5em, style=sameline]	
	\begin{lhp}{nf}{NF}{nfunk:beispiel12}
		\item [Name:] Systemsprache
		\item [Beschreibung:] Die Interfacesprache ist Deutsch.
		\item [Motivation:] Synchronisation des Verständnisses von Teammitgliedern mit unterschiedlichen kulturellen Hintergründen.
		\item [Erfüllungskriterium:] 
	\end{lhp}
\end{description}

\section{Rechtliche und standardsbezogene Anforderungen}


\begin{description}[leftmargin=5em, style=sameline]	
	\begin{lhp}{nf}{NF}{nfunk:beispiel13}
		\item [Name:] Nicht rechtliche Anforderungen
		\item [Beschreibung:] Keine relevanten rechtlichen Anforderungen bekannt.
		\item [Motivation:] Siehe \ref{fa:fortentwicklung}.
		\item [Erfüllungskriterium:] Nicht anwendbar.
	\end{lhp}
\end{description}
