\chapter{Projekttreiber}

\section{Projektziel}

Im Rahmen des Software-Entwicklungs-Projekts/Modellierungspraktikums {\the\year} soll ein einfach zu bedienendes Client-Server-System zum Spielen von Exploding Kittens über ein Netzwerk implementiert werden. Die Benutzeroberfläche soll intuitiv bedienbar sein.

\section{Stakeholders}

\newcounter{sh}\setcounter{sh}{10}

\begin{description}[leftmargin=5em, style=sameline]
	
	\begin{lhp}{sh}{SH}{sh:Spieler}
		\item [Name:] Spieler
		\item [Beschreibung:] Menschliche Spieler.
		\item [Ziele/Aufgaben:] Das Spiel zu spielen.
	\end{lhp}
	
	\begin{lhp}{sh}{SH}{bsh:Spieler}
		\item [Name:] Eltern
		\item [Beschreibung:] Eltern minderjähriger Spieler.
		\item [Ziele/Aufgaben:] Wollen Spielzeit ihrer Kinder begrenzen und Zugriff auf sensible Inhalte kontrollieren.
	\end{lhp}
	
	\begin{lhp}{sh}{SH}{bsh:gesetzgeber}
		\item [Name:] Gesetzgeber
		\item [Beschreibung:] Das Amt für Jugend und Familie.
		\item [Ziele/Aufgaben:] Die Rechte der Spieler schützen und gewähren indem er Gesetze erstellt. 
	\end{lhp}
	
	\begin{lhp}{sh}{SH}{bsh:investor}
		\item [Name:] Investoren (nur für Beispielzwecken)
		\item [Beschreibung:] Parteien, die das Finanzmittel für die Entwicklung des Systems bereitstellen.
		\item [Ziele/Aufgaben:] Gewinn zu erzielen, indem das System an Endverbraucher verkauft wird.
	\end{lhp}
	
	\begin{lhp}{sh}{SH}{bsh:betreuer}
		\item [Name:] Betreuer
		\item [Beschreibung:] HiWis, die SEP/MP Projektgruppen betreuen.
		\item [Ziele/Aufgaben:] Die Spielentwicklung durch die Studenten betreuen
	\end{lhp}
	
	\begin{lhp}{sh}{SH}{bsh:prof}
		\item [Name:] Prof. Dr. Achim Ebert
		\item [Beschreibung:] Leiter des Software-Entwicklunsprojekt
		\item [Ziele/Aufgaben:]  Leitung der Lehrverantstaltung
	\end{lhp}
		
\end{description}

\section{Aktuelle Lage}

Aktuell gibt es nur eine physische Form des Spiels und eine App für mobile Endgeräte. Da es zur Zeit keine Version des Spiels für Desktop-Computer gibt entsteht die Problematik, dass ein potenzieller Spieler ohne ein entsprechendes Handy keine Möglichkeit besitzt am Spiel teilzunehmen. Die physischen Spielkarten weisen außerdem das Problem auf, dass sie sich bei häufigem Gebrauch abnutzen oder kaputt gehen. \\Dieses Projekt soll solchen Spielern ermöglichen ohne Handy online miteinander zu spielen und einer überhöhten Kartenabnutzung vorzubeugen. Es leistet somit auch einen Beitrag zur Müllvermeidung. \\Eltern können das Spielverhalten ihrer Kinder in dieser Version genauer kontrollieren als bei den oben Genannten Varianten. \\Investoren profitieren durch die Spielverkäufe an so neu gewonnene Spieler. 