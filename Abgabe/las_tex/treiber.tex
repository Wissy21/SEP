\chapter{Projekttreiber}

\section{Projektziel}

Im Rahmen des Software-Entwicklungs-Projekts/Modellierungspraktikums {\the\year} soll ein einfach zu bedienendes Client-Server-System zum Spielen von Exploding Kittens über ein Netzwerk implementiert werden. Die Benutzeroberfläche soll intuitiv bedienbar sein.

\section{Stakeholders}

\newcounter{sh}\setcounter{sh}{10}

\begin{description}[leftmargin=5em, style=sameline]
	
	\begin{lhp}{sh}{SH}{sh:Spieler}
		\item [Name:] Spieler
		\item [Beschreibung:] Menschliche Spieler.
		\item [Ziele/Aufgaben:] Das Spiel zu spielen.
	\end{lhp}
	
	\begin{lhp}{sh}{SH}{bsh:Spieler}
		\item [Name:] Eltern
		\item [Beschreibung:] Eltern minderjähriger Spieler.
		\item [Ziele/Aufgaben:] Um die Spieler zu kümmern, indem Eltern Spielzeit begrenzen wollen und zugriff auf sensible Inhalte begrenzen.
	\end{lhp}
	
	\begin{lhp}{sh}{SH}{bsh:gesetzgeber}
		\item [Name:] Gesetzgeber
		\item [Beschreibung:] Das Amt für Jugend und Familie.
		\item [Ziele/Aufgaben:] Die Rechte der Spieler zu schützen und zu gewähren, indem er Gesetze erstellt.
	\end{lhp}
	
	\begin{lhp}{sh}{SH}{bsh:investor}
		\item [Name:] Investoren (nur für Beispielzwecken)
		\item [Beschreibung:] Parteien, die das Finanzmittel für die Entwicklung des Systems bereitstellen.
		\item [Ziele/Aufgaben:] Gewinn zu ermitteln, indem das System an Endverbraucher verkauft wird.
	\end{lhp}
	
	\begin{lhp}{sh}{SH}{bsh:betreuer}
		\item [Name:] Betreuer
		\item [Beschreibung:] HiWis, die SEP/MP Projektgruppen betreuen.
		\item [Ziele/Aufgaben:] Das Entwicklungsprozess zu betreuen, zu überwachen und teilweise zu steuern als auch die Arbeit der Projektgruppen abzunehmen sowie den Studenten im Prozess Hilfe zur Verfügung zu stellen. 
	\end{lhp}
	
	\begin{lhp}{sh}{SH}{bsh:prof}
		\item [Name:] Prof. Dr. Achim Ebert
		\item [Beschreibung:] 
		\item [Ziele/Aufgaben:]  
	\end{lhp}
		
\end{description}

\section{Aktuelle Lage}

Aktuell gibt es nur eine physische Form des Spiels und eine App auf mobilen Endgeräten. Da es keinen Version des Spiels auf Computern gibt entsteht die Problematik, dass ein potenzieller Spieler ohne ein fähiges Handy keine Möglichkeit hat das Spiel mit Freunden die er nicht besuchen kann zu spielen. Die physischen Spielkarten haben zusätzlich das Problem, das sie sich im Laufe der Zeit abnutzen oder sogar kaputt gehen können. \\Das Projekt wird den Spielern ermöglichen ohne Handy online miteinander spielen zu können. Außerdem wird der Zustand des Projekt über Zeit nicht schlechter, wie es bei den Karten der Fall ist.\\Die Eltern profitieren davon, dass sie das Spielverhalten ihrer Kinder genauer kontrollieren können als mit den anderen Spielmöglichkeiten. \\Investoren profitieren durch eine neue Einnahmequelle mit vielen potenziellen neuen Spielern.