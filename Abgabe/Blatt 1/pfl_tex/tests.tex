\chapter{Systemtestfälle}

Hier sollen verschiedene Szenarien beschrieben werden, mithilfe deren Sie später Systemtests ausführen und die erwarteten Ergebnisse darstellen.

\newcounter{tf}\setcounter{tf}{10}

\begin{description}[leftmargin=5em, style=sameline]

\begin{lhp}{tf}{TF}{tests:anmelden}
	\item [Name:] Spieler anmelden.
	\item [Motivation:] Testet, ob die Anmeldung im System korrekt funktioniert.
	\item [Szenarien:] \hfill
		\begin{enumerate}
			\item \textit{Zugriffsdaten sind vorhanden und richtig} \\ $\implies$ Spieler wird in die Lobby bewegt.
			\item \textit{Benutzername ist registriert, Passwort ist falsch} \\ $\implies$ Fehlermeldung wird angezeigt.
			\item \textit{Benutzername ist nicht registriert} \\ $\implies$ Fehlermeldung wird angezeigt.
		\end{enumerate}
	\item [Relevante Systemfunktionen:] \ref{funk:zugriff}
	\item [Relevante Use Cases:] \ref{uc:anmeld}
\end{lhp}

\end{description}

\begin{description}[leftmargin=5em, style=sameline]

\begin{lhp}{tf}{TF}{tests:registrieren}
	\item [Name:] Spieler registrieren.
	\item [Motivation:] Testet, ob die Registrierung in dem System korrekt funktioniert.
	\item [Szenarien:] \hfill
		\begin{enumerate}
			\item \textit{Benutzername sind vorhanden und richtig} \\ $\implies$ Ein Account mit den eingetragenen Daten wird erzeugt und der Spieler wird in die Lobby bewegt.
			\item \textit{Benutzername ist schon genutzt} \\ $\implies$ Fehlermeldung wird angezeigt.
			\item \textit{Email ist schon genutzt} \\ $\implies$ Fehlermeldung wird angezeigt.
		\end{enumerate}
	\item [Relevante Systemfunktionen:] \ref{funk:zugriff}
	\item [Relevante Use Cases:] \ref{uc:registrieren}
\end{lhp}

\end{description}

\begin{description}[leftmargin=5em, style=sameline]

\begin{lhp}{tf}{TF}{tests:account bearbeiten}
	\item [Name:] Account bearbeiten.
	\item [Motivation:] Testet, ob das Bearbeiten des Accounts in dem System korrekt funktioniert.
	\item [Szenarien:] \hfill
		\begin{enumerate}
			\item \textit{Neuer Benutzername nicht vorhanden} \\ $\implies$ Benutzername wird erfolgreich geändert.
			\item \textit{Neuer Benutzername  bereits vorhanden} \\ $\implies$ Fehlermeldung wird angezeigt.
			\item \textit{Neues Passwort bearbeiten, altes richtig} \\ $\implies$ Passwort wird erfolgreich geändert.
			\item \textit{Neues Passwort bearbeiten, altes falsch} \\ $\implies$ Fehlermeldung wird angezeigt.
		\end{enumerate}
	\item [Relevante Systemfunktionen:] \ref{funk:accountverw}
	\item [Relevante Use Cases:] \ref{uc:namechange}, \ref{uc:pwchange}, \ref{uc:loeschen}
\end{lhp}

\end{description}

\begin{description}[leftmargin=5em, style=sameline]

\begin{lhp}{tf}{TF}{tests:accountlöschen}
	\item [Name:] Account löschen.
	\item [Motivation:] Testet, ob das Löschen des Accounts in dem System korrekt funktioniert.
	\item [Szenarien:] \hfill
		\begin{enumerate}
			\item \textit{Passwort richtig} \\ $\implies$ Spieler wird in den Anmeldebildschirm bewegt, Account wird gelöscht.
			\item \textit{Passwort falsch} \\ $\implies$ Fehlermeldung wird angezeigt.
	
		\end{enumerate}
	\item [Relevante Systemfunktionen:] \ref{funk:zugriff}
	\item [Relevante Use Cases:] \ref{uc:loeschen}
\end{lhp}

\end{description}

\begin{description}[leftmargin=5em, style=sameline]

\begin{lhp}{tf}{TF}{tests:Spielraum}
	\item [Name:] Spielraum erstellen,bearbeiten oder löschen.
	\item [Motivation:] Testet, ob das Erstellen,das Bearbeiten oder das Löschen eines Spielraums im System korrekt funktioniert.
	\item [Szenarien:] \hfill
		\begin{enumerate}
			\item \textit{Spielraum erstellt} \\ $\implies$ Spieler wird im erstellte Spielraum bewegt.
			\item \textit{Spielraum erstellt funktioniert nicht} \\ $\implies$ Fehlermeldung wird angezeigt und kein Spielraum wird erstellt.
			\item \textit{Spielraum bearbeiten} \\ $\implies$ Spielraum Eigenschaften werden geändert.
			\item \textit{Spielraum bearbeiten funktioniert nicht} \\ $\implies$ Fehlermeldung wird angezeigt.
			\item \textit{Spielraum löschen} \\ $\implies$ Spieler soll das Löschen bestätigen.
							         \\ $\implies$ Spieler wird im Spiel Menü bewegt.
\\ $\implies$ Spielraum wird gelöscht.
			\item \textit{Spielraum löschen funktioniert nicht} \\ $\implies$ Fehlermeldung wird angezeigt.
\\ $\implies$ Spielraum wird nicht gelöst.
		\end{enumerate}
	\item [Relevante Systemfunktionen:] \ref{funk:spielraum}
	\item [Relevante Use Cases:] \ref{uc:spielen}
\end{lhp}

\end{description}

\begin{description}[leftmargin=5em, style=sameline]

\begin{lhp}{tf}{TF}{tests:Bots }
	\item [Name:] Bots.
	\item [Motivation:] Testet, ob das Hinzufügen bzw. das Entfernen der Bots korrekt funktioniert.
	\item [Szenarien:] \hfill
		\begin{enumerate}
			\item \textit{Bots hinzufügen, Platz vorhanden} \\ $\implies$ Bots werden im Spielraum bewegt.
			\item \textit{Bots hinzufügen, kein Platz vorhanden} \\ $\implies$ Fehlermeldung wird angezeigt.
			\item \textit{Bots entfernen, Bot im Raum} \\ $\implies$ Bots werden entfernt.
			\item \textit{Bots entfernen, kein Bot im Raum} \\ $\implies$ Fehlermeldung wird angezeigt.
		\end{enumerate}
	\item [Relevante Systemfunktionen:] \ref{funk:bots}
	\item [Relevante Use Cases:] \ref{uc:intelligentbots}
\end{lhp}

\end{description}

\begin{description}[leftmargin=5em, style=sameline]

\begin{lhp}{tf}{TF}{tests:Spiel}
	\item [Name:] Spiel.
	\item [Motivation:] Testet, ob das Starten bzw. das Ende des Spiels korrekt funktioniert.
	\item [Szenarien:] \hfill
		\begin{enumerate}
			\item \textit{Spiel starten, genug Spieler} \\ $\implies$ Spiel starten.
														\\ $\implies$ Spieler werden im Spielraum bewegt.
			\item \textit{Spiel starten, nicht genug Spieler} \\ $\implies$ Fehlermeldung wird angezeigt.
			\item \textit{Spiel beenden funktioniert} \\ $\implies$ Spiel wird  beendet.
													  \\ $\implies$ Spieler werden  in Lobby bewegt.
			\item \textit{Spiel beenden funktioniert nicht} \\ $\implies$ Fehlermeldung wird angezeigt.
		\end{enumerate}
	\item [Relevante Systemfunktionen:] \ref{funk:spielverw}
	\item [Relevante Use Cases:] \ref{uc:spielen}
\end{lhp}

\end{description}


\begin{description}[leftmargin=5em, style=sameline]

\begin{lhp}{tf}{TF}{tests:Bestenliste}
	\item [Name:] Bestenliste.
	\item [Motivation:] Testet, ob das Anzeigen der Bestenliste  korrekt funktioniert.
	\item [Szenarien:] \hfill
		\begin{enumerate}
			\item \textit{Bestenliste anzeigen, mindestens ein Spiel abgeschlossen} \\ $\implies$ Bestenliste anzeigen.
			\item \textit{Bestenliste anzeigen, bisher niemand gespielt} \\ $\implies$ Fehlermeldung wird angezeigt.
		\end{enumerate}
	\item [Relevante Systemfunktionen:] \ref{funk:bestenliste}
	\item [Relevante Use Cases:] \ref{uc:bestenlistesehen}
\end{lhp}

\end{description}


\begin{description}[leftmargin=5em, style=sameline]

\begin{lhp}{tf}{TF}{tests:chat}
	\item [Name:] Chatroom.
	\item [Motivation:] Testet, ob das Empfangen bzw. das Senden von Nachrichten korrekt funktioniert.
	\item [Szenarien:] \hfill
		\begin{enumerate}
			\item \textit{Nachricht schicken, mit Inhalt} \\ $\implies$ Nachricht wird verschickt und erscheint im Chatfenster aller Spieler.
			\item \textit{Nachricht schicken, kein Inhalt} \\ $\implies$ Fehlermeldung wird angezeigt.
										\\ $\implies$Keine Nachricht wird verschickt.
		\end{enumerate}
	\item [Relevante Systemfunktionen:] \ref{funk:chat}
	\item [Relevante Use Cases:] \ref{uc:chat}
\end{lhp}

\end{description}


\begin{description}[leftmargin=5em, style=sameline]

\begin{lhp}{tf}{TF}{tests: Spieler Punkte}
	\item [Name:] Spieler Punkte.
	\item [Motivation:] Testet, ob die Spieler Punkte übertragen werden.
	\item [Szenarien:] \hfill
		\begin{enumerate}
			\item \textit{Das Spiel ist fertig. Kein Spielabbruch.} \\ $\implies$ Verdiente Punkte werden zu den vorherigen Punkte addiert.
			\item \textit{Das Spiel ist fertig. Spielabbruch.} \\ $\implies$ Keine Punkte werden zu den vorherigen Punkte addiert.
		\end{enumerate}
	\item [Relevante Systemfunktionen:] \ref{funk:spielverw}
	\item [Relevante Use Cases:] \ref{uc:ausscheiden}
\end{lhp}

\end{description}

\begin{description}[leftmargin=5em, style=sameline]

\begin{lhp}{tf}{TF}{tests:spielstart}
	\item [Name:] Spielstart
	\item [Motivation:] Testet, ob der Spielstart regulär funktioniert
	\item [Szenarien:] \hfill
		\begin{enumerate}
			\item \textit{Spiel wird erfolgreich gestartet.} \\ $\implies$ Jeder Spieler hat eine Entschärfung auf der Hand.
															 \\ $\implies$ Jeder Spieler hat insgesamt 8 Handkarten.
															 \\ $\implies$ Kein Spieler hat ein Exploding Kitten auf der Hand.
			\item \textit{Spiel wird nicht gestartet.} \\ $\implies$ Der Spieler erhält eine Fehlermeldung.
		\end{enumerate}
	\item [Relevante Systemfunktionen:] \ref{funk:spielverw}
	\item [Relevante Use Cases:] \ref{uc:spielstart}
\end{lhp}

\end{description}

\begin{description}[leftmargin=5em, style=sameline]

\begin{lhp}{tf}{TF}{tests:Kartelegen}
	\item [Name:] Karten legen.
	\item [Motivation:] Testet, ob eine Karte gelegt werden kann
	\item [Szenarien:] \hfill
		\begin{enumerate}
			\item \textit{Spieler ist am Zug und legt eine regelkonforme Karte.} \\ $\implies$ Karteneffekt wird ausgeführt und Karte wird auch den Ablagestapel gelegt.
			\item \textit{Spieler ist am Zug und legt eine nicht regelkonforme Karte.} \\ $\implies$ Es wird keine Karte abgelegt und der Spieler erhält eine Fehlermeldung.
		\end{enumerate}
	\item [Relevante Systemfunktionen:] \ref{funk:spielverw}
	\item [Relevante Use Cases:] \ref{uc:kartelegen}
\end{lhp}

\end{description}

\begin{description}[leftmargin=5em, style=sameline]

\begin{lhp}{tf}{TF}{tests:zugbeenden}
	\item [Name:] Zug beenden.
	\item [Motivation:] Testet, ob der Zug beendet werden kann
	\item [Szenarien:] \hfill
		\begin{enumerate}
			\item \textit{Spieler ist am Zug.} \\ $\implies$ Karteneffekt wird ausgeführt und Karte wird auch den Ablagestapel gelegt.
			\item \textit{Spieler ist nicht am Zug.} \\ $\implies$ Der Spieler erhält eine Fehlermeldung.
		\end{enumerate}
	\item [Relevante Systemfunktionen:] \ref{funk:spielverw}
	\item [Relevante Use Cases:] \ref{uc:zugbeenden}
\end{lhp}

\end{description}

\begin{description}[leftmargin=5em, style=sameline]

\begin{lhp}{tf}{TF}{tests:entschärfen}
	\item [Name:] Entschärfen
	\item [Motivation:] Testet, ob die Entschärfung funktioniert
	\item [Szenarien:] \hfill
		\begin{enumerate}
			\item \textit{Spieler hat ein Exploding Kitten und eine Entschärfung auf der Hand.} \\ $\implies$  Legt Entschärfung auf den Ablagestapel und Exploding Kitten in den Spielstapel. Spieler bleibt im Spiel.
			\item \textit{Spieler hat ein Exploding Kitten aber keine Entschärfung auf der Hand.} \\ $\implies$  Legt alle Karten des Spielers auf den Ablagestapel. Spieler scheidet aus dem Spiel aus.		
		\end{enumerate}
	\item [Relevante Systemfunktionen:] \ref{funk:spielverw}
	\item [Relevante Use Cases:] \ref{uc:entschärfen}, \ref{uc:ausscheiden}
\end{lhp}

\end{description}

\begin{description}[leftmargin=5em, style=sameline]

\begin{lhp}{tf}{TF}{tests:gewinnen}
	\item [Name:] Gewinnen
	\item [Motivation:] Testet, ob man am Ende gewinnt
	\item [Szenarien:] \hfill
		\begin{enumerate}
			\item \textit{Nach dem ausscheiden eines Spielers sind noch mehr als ein Spieler übrig.} \\ $\implies$  Das Spiel geht weiter.
			\item \textit{Nach dem ausscheiden eines Spielers ist nur noch ein Spieler übrig.} \\ $\implies$  Der letzte	verbleibende Spieler hat gewonnen und bekommt Punkte für die Bestenliste. Alle Spieler kehren zurück in die Lobby.	
		\end{enumerate}
	\item [Relevante Systemfunktionen:] \ref{funk:spielverw}
	\item [Relevante Use Cases:] \ref{uc:ausscheiden}, \ref{uc:spielende}
\end{lhp}

\end{description}
























